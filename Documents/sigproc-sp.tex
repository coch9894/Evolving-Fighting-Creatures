% THIS IS SIGPROC-SP.TEX - VERSION 3.1
% WORKS WITH V3.2SP OF ACM_PROC_ARTICLE-SP.CLS
% APRIL 2009
%
% It is an example file showing how to use the 'acm_proc_article-sp.cls' V3.2SP
% LaTeX2e document class file for Conference Proceedings submissions.
% ----------------------------------------------------------------------------------------------------------------
% This .tex file (and associated .cls V3.2SP) *DOES NOT* produce:
%       1) The Permission Statement
%       2) The Conference (location) Info information
%       3) The Copyright Line with ACM data
%       4) Page numbering
% ---------------------------------------------------------------------------------------------------------------
% It is an example which *does* use the .bib file (from which the .bbl file
% is produced).
% REMEMBER HOWEVER: After having produced the .bbl file,
% and prior to final submission,
% you need to 'insert'  your .bbl file into your source .tex file so as to provide
% ONE 'self-contained' source file.
%
% Questions regarding SIGS should be sent to
% Adrienne Griscti ---> griscti@acm.org
%
% Questions/suggestions regarding the guidelines, .tex and .cls files, etc. to
% Gerald Murray ---> murray@hq.acm.org
%
% For tracking purposes - this is V3.1SP - APRIL 2009

\documentclass{acm_proc_article-sp}

\begin{document}

\title{Evolving Fighting Creatures: A Look into Fitness and Competitive Coevolution}
%
% You need the command \numberofauthors to handle the 'placement
% and alignment' of the authors beneath the title.
%
% For aesthetic reasons, we recommend 'three authors at a time'
% i.e. three 'name/affiliation blocks' be placed beneath the title.
%
% NOTE: You are NOT restricted in how many 'rows' of
% "name/affiliations" may appear. We just ask that you restrict
% the number of 'columns' to three.
%
% Because of the available 'opening page real-estate'
% we ask you to refrain from putting more than six authors
% (two rows with three columns) beneath the article title.
% More than six makes the first-page appear very cluttered indeed.
%
% Use the \alignauthor commands to handle the names
% and affiliations for an 'aesthetic maximum' of six authors.
% Add names, affiliations, addresses for
% the seventh etc. author(s) as the argument for the
% \additionalauthors command.
% These 'additional authors' will be output/set for you
% without further effort on your part as the last section in
% the body of your article BEFORE References or any Appendices.

\numberofauthors{2} %  in this sample file, there are a *total*
% of EIGHT authors. SIX appear on the 'first-page' (for formatting
% reasons) and the remaining two appear in the \additionalauthors section.
%
\author{
% You can go ahead and credit any number of authors here,
% e.g. one 'row of three' or two rows (consisting of one row of three
% and a second row of one, two or three).
%
% The command \alignauthor (no curly braces needed) should
% precede each author name, affiliation/snail-mail address and
% e-mail address. Additionally, tag each line of
% affiliation/address with \affaddr, and tag the
% e-mail address with \email.
%
% 1st. author
\alignauthor
Alex Cochrane\\
       \affaddr{University of Idaho}\\
       \email{coch9894@vandals.uidaho.edu}
% 2nd. author
\alignauthor
Jordan Leithart\\
       \affaddr{University of Idaho}\\
       \email{leit7193@vandals.uidaho.edu}
}
% There's nothing stopping you putting the seventh, eighth, etc.
% author on the opening page (as the 'third row') but we ask,
% for aesthetic reasons that you place these 'additional authors'
% in the \additional authors block, viz.

% Just remember to make sure that the TOTAL number of authors
% is the number that will appear on the first page PLUS the
% number that will appear in the \additionalauthors section.

\maketitle
\begin{abstract}

An essential part of any genetic program is the use of a well defined fitness function that produces the desired outputs. For competitive coevolution this does not change however, the ability to view the affects of different fitness functions on two simultaneously evolving populations can be seen through competition. Through competition, the value of a good fitness function will become apparent from the winner of the competition. We propose that it is possible to see the affects of different fitness functions through control of an individuals fitness which then can be normalized to compare to other individuals fitnesses in the population.

HOW WE DID IT

RESULTS

\end{abstract}

% A category with the (minimum) three required fields
%\category{H.4}{Information Systems Applications}{Miscellaneous}
%A category including the fourth, optional field follows...
%\category{D.2.8}{Software Engineering}{Metrics}[complexity measures, performance measures]

%\terms{}

\keywords{Genetic Programming, Coevolution, Competitive Coevolution, Evolutionary Computation, Red Queen Effect, Fitness Function} % NOT required for Proceedings

\section{Introduction} % Include Papers here

STUFF

PAPERS STUFF

\section{The Experiment} % BODY

The simplest way to test our hypothesis was to build a genetic program that could be used in conjuncture with a graphics engine so our end results could be shown and quantified. To do so we built a genetic program that would evolve the instructions for an individual. To evaluate these individuals with different fitness functions it seemed best to use the idea of training two populations on separate fitness functions and then testing them against each other at regular intervals to get a grasp on the validity of each fitness function. After each generation, the best individual from \textit{either} population was saved into a separate population as a control group. When evolution ended, the best individual was tested against this control group to check for the Red Queen Effect. This allowed us to observe if we were creating a overall best solution or if we were cycling through a few strategies.

\subsection{Individuals} % What makes an individual
Our individuals were made up of common and new behaviours that we needed to model our creatures. The following list of Non-Terminals and Terminals is all we used to make up our individual. Evolution was the key to setting them in the correct order and will be covered in the next section.
\begin{itemize}
\item Non-Terminals
	\begin{itemize}
	\item Prog2 - When called during evaluation, pushes its left child, then its right child onto the correct player stack.
	\item Prog3 - When called during evaluation, pushes its left child, middle child, then its right child onto the correct player stack.
	\end{itemize}
\item Terminals
	\begin{itemize}
	\item Move - When called during evaluation, moves the correct players x position by cos(this->direction)*speed + 0.5 where the direction is where the player is facing. Also moves the correct players y position by sin(this->direction)*speed + 0.5.
	\item Turn Left - When called during evaluation, computes this->direction += BASE\_ANGLE which changes the direction the player is facing.
	\item Turn Right - When called during evaluation, computes this->direction -= BASE\_ANGLE which changes the direction the player is facing.
	\item Aim - When called during evaluation, computes the angle to turn by using the current position of both players and the arc-tangent. In other words, we set the angle to turn to be atan( (y2-y1)/(x2-x1) ).
	\item Shoot - When called during evaluation, adds a new bullet to the environment list containing environment variables.
	\end{itemize}
\end{itemize}

\subsection{Genetic Program} % GP

\begin{table*}
\centering
\caption{Evolutionary Characteristics}
\begin{tabular}{|l|p{4in}|}
\hline
Algorithm & \\
\hline
Population size & \\
\hline
Selection method & \\
\hline
Elitism & \\
\hline
Crossover method & \\
\hline
Crossover rate & \\
\hline
Mutation method & \\
\hline
Operator/non-terminal set & \\
\hline
Terminal set & \\
\hline
Fitness function & \\
\hline
\end{tabular}
\end{table*}

SELECTION

CROSSOVER

MUTATION

\subsection{Fitness \& Other Algorithms} % Fitness etc.

To keep the fitnesses simple we decided to keep track the number of times an individual hit the opponent and the number of times that an opponent hit the individual. These were the numbers that our fitness functions would try to control. For example we might want to try and maximize the number of successful hits of the opponent and maximize the number of times the opponent hits the individual. In theory this doesn't sound like that great of a strategy but maybe against some other fitness function it would be great.

When we would like to compare the different functions however we needed some way to standardize these fitnesses. We came up with what we viewed as a perfect fitness and created a normalizing function to use the two prior values to represent it. Therefore our fitness functions are really controlling methods and strategies for getting the highest normalized fitness. The following is pseudo code for our normalization function:
\begin{verbatim}
    if(this->numSuccess == this->numFail)
    {
        this->fitness = 1;
    }
    else
    {
        if(this->numFail == 0){
            this->numFail += 0.01;
        }
        this->fitness = numSuccess/numFail;
    }
\end{verbatim}
Because we decided to use a division of Success/Failures, we had to account for 0 in the denominator. Therefore if the Successes=Failures we just set our "stalemate" value or "average" outcome at 1. Therefore anything over 1 is assumed to be above average during the division. If the denominator is 0 and the numerator is not, we take this as an exceptionally good individual and wish to divide by 0.01 which essentially multiplies our answer by 100.

\subsubsection{Turning Logic} % Our Turn_Left and Turn_Right Logic

When training or testing to obtain fitness, it is unknown which side of the board the individual will be on or if they will always be on the same side of the board. To compensate for this, when a player is designated to be Player1, a turn\_left action will cause them to rotate their facing direction in a positive direction and a turn\_right action will cause them to rotate their facing direction in a negative direction. However, when a player is designated as Player2, a turn\_left action will cause them to rotate their facing direction in a negative direction and a turn\_right action will cause them to rotate their facing direction in a positive direction. This removes any chance of an individual evolving a strategy for only one side of the board. This will help create a universal strategy.

\section{Results} % Results



\section{Conclusions} % Conclusions



%
% The following two commands are all you need in the
% initial runs of your .tex file to
% produce the bibliography for the citations in your paper.
\bibliographystyle{abbrv}
\bibliography{sigproc}  % sigproc.bib is the name of the Bibliography in this case
% You must have a proper ".bib" file
%  and remember to run:
% latex bibtex latex latex
% to resolve all references
%
% ACM needs 'a single self-contained file'!
%
%APPENDICES are optional
%\balancecolumns



\appendix
%Appendix A
\section{Headings in Appendices}
The rules about hierarchical headings discussed above for
the body of the article are different in the appendices.
In the \textbf{appendix} environment, the command
\textbf{section} is used to
indicate the start of each Appendix, with alphabetic order
designation (i.e. the first is A, the second B, etc.) and
a title (if you include one).  So, if you need
hierarchical structure
\textit{within} an Appendix, start with \textbf{subsection} as the
highest level. Here is an outline of the body of this
document in Appendix-appropriate form:
\subsection{Introduction}
\subsection{The Body of the Paper}
\subsubsection{Type Changes and  Special Characters}
\subsubsection{Math Equations}
\paragraph{Inline (In-text) Equations}
\paragraph{Display Equations}
\subsubsection{Citations}
\subsubsection{Tables}
\subsubsection{Figures}
\subsubsection{Theorem-like Constructs}
\subsubsection*{A Caveat for the \TeX\ Expert}
\subsection{Conclusions}
\subsection{Acknowledgments}
\subsection{Additional Authors}
This section is inserted by \LaTeX; you do not insert it.
You just add the names and information in the
\texttt{{\char'134}additionalauthors} command at the start
of the document.
\subsection{References}
Generated by bibtex from your ~.bib file.  Run latex,
then bibtex, then latex twice (to resolve references)
to create the ~.bbl file.  Insert that ~.bbl file into
the .tex source file and comment out
the command \texttt{{\char'134}thebibliography}.
% This next section command marks the start of
% Appendix B, and does not continue the present hierarchy
\section{More Help for the Hardy}
The acm\_proc\_article-sp document class file itself is chock-full of succinct
and helpful comments.  If you consider yourself a moderately
experienced to expert user of \LaTeX, you may find reading
it useful but please remember not to change it.
\balancecolumns
% That's all folks!
\end{document}
